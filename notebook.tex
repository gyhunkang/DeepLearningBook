
% Default to the notebook output style

    


% Inherit from the specified cell style.




    
\documentclass[11pt]{article}

    
    
    \usepackage[T1]{fontenc}
    % Nicer default font (+ math font) than Computer Modern for most use cases
    \usepackage{mathpazo}

    % Basic figure setup, for now with no caption control since it's done
    % automatically by Pandoc (which extracts ![](path) syntax from Markdown).
    \usepackage{graphicx}
    % We will generate all images so they have a width \maxwidth. This means
    % that they will get their normal width if they fit onto the page, but
    % are scaled down if they would overflow the margins.
    \makeatletter
    \def\maxwidth{\ifdim\Gin@nat@width>\linewidth\linewidth
    \else\Gin@nat@width\fi}
    \makeatother
    \let\Oldincludegraphics\includegraphics
    % Set max figure width to be 80% of text width, for now hardcoded.
    \renewcommand{\includegraphics}[1]{\Oldincludegraphics[width=.8\maxwidth]{#1}}
    % Ensure that by default, figures have no caption (until we provide a
    % proper Figure object with a Caption API and a way to capture that
    % in the conversion process - todo).
    \usepackage{caption}
    \DeclareCaptionLabelFormat{nolabel}{}
    \captionsetup{labelformat=nolabel}

    \usepackage{adjustbox} % Used to constrain images to a maximum size 
    \usepackage{xcolor} % Allow colors to be defined
    \usepackage{enumerate} % Needed for markdown enumerations to work
    \usepackage{geometry} % Used to adjust the document margins
    \usepackage{amsmath} % Equations
    \usepackage{amssymb} % Equations
    \usepackage{textcomp} % defines textquotesingle
    % Hack from http://tex.stackexchange.com/a/47451/13684:
    \AtBeginDocument{%
        \def\PYZsq{\textquotesingle}% Upright quotes in Pygmentized code
    }
    \usepackage{upquote} % Upright quotes for verbatim code
    \usepackage{eurosym} % defines \euro
    \usepackage[mathletters]{ucs} % Extended unicode (utf-8) support
    \usepackage[utf8x]{inputenc} % Allow utf-8 characters in the tex document
    \usepackage{fancyvrb} % verbatim replacement that allows latex
    \usepackage{grffile} % extends the file name processing of package graphics 
                         % to support a larger range 
    % The hyperref package gives us a pdf with properly built
    % internal navigation ('pdf bookmarks' for the table of contents,
    % internal cross-reference links, web links for URLs, etc.)
    \usepackage{hyperref}
    \usepackage{longtable} % longtable support required by pandoc >1.10
    \usepackage{booktabs}  % table support for pandoc > 1.12.2
    \usepackage[inline]{enumitem} % IRkernel/repr support (it uses the enumerate* environment)
    \usepackage[normalem]{ulem} % ulem is needed to support strikethroughs (\sout)
                                % normalem makes italics be italics, not underlines
    

    
    
    % Colors for the hyperref package
    \definecolor{urlcolor}{rgb}{0,.145,.698}
    \definecolor{linkcolor}{rgb}{.71,0.21,0.01}
    \definecolor{citecolor}{rgb}{.12,.54,.11}

    % ANSI colors
    \definecolor{ansi-black}{HTML}{3E424D}
    \definecolor{ansi-black-intense}{HTML}{282C36}
    \definecolor{ansi-red}{HTML}{E75C58}
    \definecolor{ansi-red-intense}{HTML}{B22B31}
    \definecolor{ansi-green}{HTML}{00A250}
    \definecolor{ansi-green-intense}{HTML}{007427}
    \definecolor{ansi-yellow}{HTML}{DDB62B}
    \definecolor{ansi-yellow-intense}{HTML}{B27D12}
    \definecolor{ansi-blue}{HTML}{208FFB}
    \definecolor{ansi-blue-intense}{HTML}{0065CA}
    \definecolor{ansi-magenta}{HTML}{D160C4}
    \definecolor{ansi-magenta-intense}{HTML}{A03196}
    \definecolor{ansi-cyan}{HTML}{60C6C8}
    \definecolor{ansi-cyan-intense}{HTML}{258F8F}
    \definecolor{ansi-white}{HTML}{C5C1B4}
    \definecolor{ansi-white-intense}{HTML}{A1A6B2}

    % commands and environments needed by pandoc snippets
    % extracted from the output of `pandoc -s`
    \providecommand{\tightlist}{%
      \setlength{\itemsep}{0pt}\setlength{\parskip}{0pt}}
    \DefineVerbatimEnvironment{Highlighting}{Verbatim}{commandchars=\\\{\}}
    % Add ',fontsize=\small' for more characters per line
    \newenvironment{Shaded}{}{}
    \newcommand{\KeywordTok}[1]{\textcolor[rgb]{0.00,0.44,0.13}{\textbf{{#1}}}}
    \newcommand{\DataTypeTok}[1]{\textcolor[rgb]{0.56,0.13,0.00}{{#1}}}
    \newcommand{\DecValTok}[1]{\textcolor[rgb]{0.25,0.63,0.44}{{#1}}}
    \newcommand{\BaseNTok}[1]{\textcolor[rgb]{0.25,0.63,0.44}{{#1}}}
    \newcommand{\FloatTok}[1]{\textcolor[rgb]{0.25,0.63,0.44}{{#1}}}
    \newcommand{\CharTok}[1]{\textcolor[rgb]{0.25,0.44,0.63}{{#1}}}
    \newcommand{\StringTok}[1]{\textcolor[rgb]{0.25,0.44,0.63}{{#1}}}
    \newcommand{\CommentTok}[1]{\textcolor[rgb]{0.38,0.63,0.69}{\textit{{#1}}}}
    \newcommand{\OtherTok}[1]{\textcolor[rgb]{0.00,0.44,0.13}{{#1}}}
    \newcommand{\AlertTok}[1]{\textcolor[rgb]{1.00,0.00,0.00}{\textbf{{#1}}}}
    \newcommand{\FunctionTok}[1]{\textcolor[rgb]{0.02,0.16,0.49}{{#1}}}
    \newcommand{\RegionMarkerTok}[1]{{#1}}
    \newcommand{\ErrorTok}[1]{\textcolor[rgb]{1.00,0.00,0.00}{\textbf{{#1}}}}
    \newcommand{\NormalTok}[1]{{#1}}
    
    % Additional commands for more recent versions of Pandoc
    \newcommand{\ConstantTok}[1]{\textcolor[rgb]{0.53,0.00,0.00}{{#1}}}
    \newcommand{\SpecialCharTok}[1]{\textcolor[rgb]{0.25,0.44,0.63}{{#1}}}
    \newcommand{\VerbatimStringTok}[1]{\textcolor[rgb]{0.25,0.44,0.63}{{#1}}}
    \newcommand{\SpecialStringTok}[1]{\textcolor[rgb]{0.73,0.40,0.53}{{#1}}}
    \newcommand{\ImportTok}[1]{{#1}}
    \newcommand{\DocumentationTok}[1]{\textcolor[rgb]{0.73,0.13,0.13}{\textit{{#1}}}}
    \newcommand{\AnnotationTok}[1]{\textcolor[rgb]{0.38,0.63,0.69}{\textbf{\textit{{#1}}}}}
    \newcommand{\CommentVarTok}[1]{\textcolor[rgb]{0.38,0.63,0.69}{\textbf{\textit{{#1}}}}}
    \newcommand{\VariableTok}[1]{\textcolor[rgb]{0.10,0.09,0.49}{{#1}}}
    \newcommand{\ControlFlowTok}[1]{\textcolor[rgb]{0.00,0.44,0.13}{\textbf{{#1}}}}
    \newcommand{\OperatorTok}[1]{\textcolor[rgb]{0.40,0.40,0.40}{{#1}}}
    \newcommand{\BuiltInTok}[1]{{#1}}
    \newcommand{\ExtensionTok}[1]{{#1}}
    \newcommand{\PreprocessorTok}[1]{\textcolor[rgb]{0.74,0.48,0.00}{{#1}}}
    \newcommand{\AttributeTok}[1]{\textcolor[rgb]{0.49,0.56,0.16}{{#1}}}
    \newcommand{\InformationTok}[1]{\textcolor[rgb]{0.38,0.63,0.69}{\textbf{\textit{{#1}}}}}
    \newcommand{\WarningTok}[1]{\textcolor[rgb]{0.38,0.63,0.69}{\textbf{\textit{{#1}}}}}
    
    
    % Define a nice break command that doesn't care if a line doesn't already
    % exist.
    \def\br{\hspace*{\fill} \\* }
    % Math Jax compatability definitions
    \def\gt{>}
    \def\lt{<}
    % Document parameters
    \title{DLBook\_ch07}
    
    
    

    % Pygments definitions
    
\makeatletter
\def\PY@reset{\let\PY@it=\relax \let\PY@bf=\relax%
    \let\PY@ul=\relax \let\PY@tc=\relax%
    \let\PY@bc=\relax \let\PY@ff=\relax}
\def\PY@tok#1{\csname PY@tok@#1\endcsname}
\def\PY@toks#1+{\ifx\relax#1\empty\else%
    \PY@tok{#1}\expandafter\PY@toks\fi}
\def\PY@do#1{\PY@bc{\PY@tc{\PY@ul{%
    \PY@it{\PY@bf{\PY@ff{#1}}}}}}}
\def\PY#1#2{\PY@reset\PY@toks#1+\relax+\PY@do{#2}}

\expandafter\def\csname PY@tok@w\endcsname{\def\PY@tc##1{\textcolor[rgb]{0.73,0.73,0.73}{##1}}}
\expandafter\def\csname PY@tok@c\endcsname{\let\PY@it=\textit\def\PY@tc##1{\textcolor[rgb]{0.25,0.50,0.50}{##1}}}
\expandafter\def\csname PY@tok@cp\endcsname{\def\PY@tc##1{\textcolor[rgb]{0.74,0.48,0.00}{##1}}}
\expandafter\def\csname PY@tok@k\endcsname{\let\PY@bf=\textbf\def\PY@tc##1{\textcolor[rgb]{0.00,0.50,0.00}{##1}}}
\expandafter\def\csname PY@tok@kp\endcsname{\def\PY@tc##1{\textcolor[rgb]{0.00,0.50,0.00}{##1}}}
\expandafter\def\csname PY@tok@kt\endcsname{\def\PY@tc##1{\textcolor[rgb]{0.69,0.00,0.25}{##1}}}
\expandafter\def\csname PY@tok@o\endcsname{\def\PY@tc##1{\textcolor[rgb]{0.40,0.40,0.40}{##1}}}
\expandafter\def\csname PY@tok@ow\endcsname{\let\PY@bf=\textbf\def\PY@tc##1{\textcolor[rgb]{0.67,0.13,1.00}{##1}}}
\expandafter\def\csname PY@tok@nb\endcsname{\def\PY@tc##1{\textcolor[rgb]{0.00,0.50,0.00}{##1}}}
\expandafter\def\csname PY@tok@nf\endcsname{\def\PY@tc##1{\textcolor[rgb]{0.00,0.00,1.00}{##1}}}
\expandafter\def\csname PY@tok@nc\endcsname{\let\PY@bf=\textbf\def\PY@tc##1{\textcolor[rgb]{0.00,0.00,1.00}{##1}}}
\expandafter\def\csname PY@tok@nn\endcsname{\let\PY@bf=\textbf\def\PY@tc##1{\textcolor[rgb]{0.00,0.00,1.00}{##1}}}
\expandafter\def\csname PY@tok@ne\endcsname{\let\PY@bf=\textbf\def\PY@tc##1{\textcolor[rgb]{0.82,0.25,0.23}{##1}}}
\expandafter\def\csname PY@tok@nv\endcsname{\def\PY@tc##1{\textcolor[rgb]{0.10,0.09,0.49}{##1}}}
\expandafter\def\csname PY@tok@no\endcsname{\def\PY@tc##1{\textcolor[rgb]{0.53,0.00,0.00}{##1}}}
\expandafter\def\csname PY@tok@nl\endcsname{\def\PY@tc##1{\textcolor[rgb]{0.63,0.63,0.00}{##1}}}
\expandafter\def\csname PY@tok@ni\endcsname{\let\PY@bf=\textbf\def\PY@tc##1{\textcolor[rgb]{0.60,0.60,0.60}{##1}}}
\expandafter\def\csname PY@tok@na\endcsname{\def\PY@tc##1{\textcolor[rgb]{0.49,0.56,0.16}{##1}}}
\expandafter\def\csname PY@tok@nt\endcsname{\let\PY@bf=\textbf\def\PY@tc##1{\textcolor[rgb]{0.00,0.50,0.00}{##1}}}
\expandafter\def\csname PY@tok@nd\endcsname{\def\PY@tc##1{\textcolor[rgb]{0.67,0.13,1.00}{##1}}}
\expandafter\def\csname PY@tok@s\endcsname{\def\PY@tc##1{\textcolor[rgb]{0.73,0.13,0.13}{##1}}}
\expandafter\def\csname PY@tok@sd\endcsname{\let\PY@it=\textit\def\PY@tc##1{\textcolor[rgb]{0.73,0.13,0.13}{##1}}}
\expandafter\def\csname PY@tok@si\endcsname{\let\PY@bf=\textbf\def\PY@tc##1{\textcolor[rgb]{0.73,0.40,0.53}{##1}}}
\expandafter\def\csname PY@tok@se\endcsname{\let\PY@bf=\textbf\def\PY@tc##1{\textcolor[rgb]{0.73,0.40,0.13}{##1}}}
\expandafter\def\csname PY@tok@sr\endcsname{\def\PY@tc##1{\textcolor[rgb]{0.73,0.40,0.53}{##1}}}
\expandafter\def\csname PY@tok@ss\endcsname{\def\PY@tc##1{\textcolor[rgb]{0.10,0.09,0.49}{##1}}}
\expandafter\def\csname PY@tok@sx\endcsname{\def\PY@tc##1{\textcolor[rgb]{0.00,0.50,0.00}{##1}}}
\expandafter\def\csname PY@tok@m\endcsname{\def\PY@tc##1{\textcolor[rgb]{0.40,0.40,0.40}{##1}}}
\expandafter\def\csname PY@tok@gh\endcsname{\let\PY@bf=\textbf\def\PY@tc##1{\textcolor[rgb]{0.00,0.00,0.50}{##1}}}
\expandafter\def\csname PY@tok@gu\endcsname{\let\PY@bf=\textbf\def\PY@tc##1{\textcolor[rgb]{0.50,0.00,0.50}{##1}}}
\expandafter\def\csname PY@tok@gd\endcsname{\def\PY@tc##1{\textcolor[rgb]{0.63,0.00,0.00}{##1}}}
\expandafter\def\csname PY@tok@gi\endcsname{\def\PY@tc##1{\textcolor[rgb]{0.00,0.63,0.00}{##1}}}
\expandafter\def\csname PY@tok@gr\endcsname{\def\PY@tc##1{\textcolor[rgb]{1.00,0.00,0.00}{##1}}}
\expandafter\def\csname PY@tok@ge\endcsname{\let\PY@it=\textit}
\expandafter\def\csname PY@tok@gs\endcsname{\let\PY@bf=\textbf}
\expandafter\def\csname PY@tok@gp\endcsname{\let\PY@bf=\textbf\def\PY@tc##1{\textcolor[rgb]{0.00,0.00,0.50}{##1}}}
\expandafter\def\csname PY@tok@go\endcsname{\def\PY@tc##1{\textcolor[rgb]{0.53,0.53,0.53}{##1}}}
\expandafter\def\csname PY@tok@gt\endcsname{\def\PY@tc##1{\textcolor[rgb]{0.00,0.27,0.87}{##1}}}
\expandafter\def\csname PY@tok@err\endcsname{\def\PY@bc##1{\setlength{\fboxsep}{0pt}\fcolorbox[rgb]{1.00,0.00,0.00}{1,1,1}{\strut ##1}}}
\expandafter\def\csname PY@tok@kc\endcsname{\let\PY@bf=\textbf\def\PY@tc##1{\textcolor[rgb]{0.00,0.50,0.00}{##1}}}
\expandafter\def\csname PY@tok@kd\endcsname{\let\PY@bf=\textbf\def\PY@tc##1{\textcolor[rgb]{0.00,0.50,0.00}{##1}}}
\expandafter\def\csname PY@tok@kn\endcsname{\let\PY@bf=\textbf\def\PY@tc##1{\textcolor[rgb]{0.00,0.50,0.00}{##1}}}
\expandafter\def\csname PY@tok@kr\endcsname{\let\PY@bf=\textbf\def\PY@tc##1{\textcolor[rgb]{0.00,0.50,0.00}{##1}}}
\expandafter\def\csname PY@tok@bp\endcsname{\def\PY@tc##1{\textcolor[rgb]{0.00,0.50,0.00}{##1}}}
\expandafter\def\csname PY@tok@fm\endcsname{\def\PY@tc##1{\textcolor[rgb]{0.00,0.00,1.00}{##1}}}
\expandafter\def\csname PY@tok@vc\endcsname{\def\PY@tc##1{\textcolor[rgb]{0.10,0.09,0.49}{##1}}}
\expandafter\def\csname PY@tok@vg\endcsname{\def\PY@tc##1{\textcolor[rgb]{0.10,0.09,0.49}{##1}}}
\expandafter\def\csname PY@tok@vi\endcsname{\def\PY@tc##1{\textcolor[rgb]{0.10,0.09,0.49}{##1}}}
\expandafter\def\csname PY@tok@vm\endcsname{\def\PY@tc##1{\textcolor[rgb]{0.10,0.09,0.49}{##1}}}
\expandafter\def\csname PY@tok@sa\endcsname{\def\PY@tc##1{\textcolor[rgb]{0.73,0.13,0.13}{##1}}}
\expandafter\def\csname PY@tok@sb\endcsname{\def\PY@tc##1{\textcolor[rgb]{0.73,0.13,0.13}{##1}}}
\expandafter\def\csname PY@tok@sc\endcsname{\def\PY@tc##1{\textcolor[rgb]{0.73,0.13,0.13}{##1}}}
\expandafter\def\csname PY@tok@dl\endcsname{\def\PY@tc##1{\textcolor[rgb]{0.73,0.13,0.13}{##1}}}
\expandafter\def\csname PY@tok@s2\endcsname{\def\PY@tc##1{\textcolor[rgb]{0.73,0.13,0.13}{##1}}}
\expandafter\def\csname PY@tok@sh\endcsname{\def\PY@tc##1{\textcolor[rgb]{0.73,0.13,0.13}{##1}}}
\expandafter\def\csname PY@tok@s1\endcsname{\def\PY@tc##1{\textcolor[rgb]{0.73,0.13,0.13}{##1}}}
\expandafter\def\csname PY@tok@mb\endcsname{\def\PY@tc##1{\textcolor[rgb]{0.40,0.40,0.40}{##1}}}
\expandafter\def\csname PY@tok@mf\endcsname{\def\PY@tc##1{\textcolor[rgb]{0.40,0.40,0.40}{##1}}}
\expandafter\def\csname PY@tok@mh\endcsname{\def\PY@tc##1{\textcolor[rgb]{0.40,0.40,0.40}{##1}}}
\expandafter\def\csname PY@tok@mi\endcsname{\def\PY@tc##1{\textcolor[rgb]{0.40,0.40,0.40}{##1}}}
\expandafter\def\csname PY@tok@il\endcsname{\def\PY@tc##1{\textcolor[rgb]{0.40,0.40,0.40}{##1}}}
\expandafter\def\csname PY@tok@mo\endcsname{\def\PY@tc##1{\textcolor[rgb]{0.40,0.40,0.40}{##1}}}
\expandafter\def\csname PY@tok@ch\endcsname{\let\PY@it=\textit\def\PY@tc##1{\textcolor[rgb]{0.25,0.50,0.50}{##1}}}
\expandafter\def\csname PY@tok@cm\endcsname{\let\PY@it=\textit\def\PY@tc##1{\textcolor[rgb]{0.25,0.50,0.50}{##1}}}
\expandafter\def\csname PY@tok@cpf\endcsname{\let\PY@it=\textit\def\PY@tc##1{\textcolor[rgb]{0.25,0.50,0.50}{##1}}}
\expandafter\def\csname PY@tok@c1\endcsname{\let\PY@it=\textit\def\PY@tc##1{\textcolor[rgb]{0.25,0.50,0.50}{##1}}}
\expandafter\def\csname PY@tok@cs\endcsname{\let\PY@it=\textit\def\PY@tc##1{\textcolor[rgb]{0.25,0.50,0.50}{##1}}}

\def\PYZbs{\char`\\}
\def\PYZus{\char`\_}
\def\PYZob{\char`\{}
\def\PYZcb{\char`\}}
\def\PYZca{\char`\^}
\def\PYZam{\char`\&}
\def\PYZlt{\char`\<}
\def\PYZgt{\char`\>}
\def\PYZsh{\char`\#}
\def\PYZpc{\char`\%}
\def\PYZdl{\char`\$}
\def\PYZhy{\char`\-}
\def\PYZsq{\char`\'}
\def\PYZdq{\char`\"}
\def\PYZti{\char`\~}
% for compatibility with earlier versions
\def\PYZat{@}
\def\PYZlb{[}
\def\PYZrb{]}
\makeatother


    % Exact colors from NB
    \definecolor{incolor}{rgb}{0.0, 0.0, 0.5}
    \definecolor{outcolor}{rgb}{0.545, 0.0, 0.0}



    
    % Prevent overflowing lines due to hard-to-break entities
    \sloppy 
    % Setup hyperref package
    \hypersetup{
      breaklinks=true,  % so long urls are correctly broken across lines
      colorlinks=true,
      urlcolor=urlcolor,
      linkcolor=linkcolor,
      citecolor=citecolor,
      }
    % Slightly bigger margins than the latex defaults
    
    \geometry{verbose,tmargin=1in,bmargin=1in,lmargin=1in,rmargin=1in}
    
    

    \begin{document}
    
    
    \maketitle
    
    

    
    \hypertarget{chapter-7}{%
\subsection{Chapter 7}\label{chapter-7}}

    \hypertarget{regularization-for-deep-learning}{%
\subsection{Regularization for Deep
Learning}\label{regularization-for-deep-learning}}

    Regularization - any modification we make to a learning algorithm that
is intended to reduce its generalization error but not its training
error.

    Most regularization strategies are based on regularizing estimators
Regularization of an estimator works by trading increased bias for
reduced variance 

    Three situations: 1. excluded the true data generating process -
corresponding to underfitting and inducing bias 1. matched the true data
generating process 1. included the generating process buy also many
other possible generating processes

    The Goal of Regularization is to take a model from the third regime into
the second regime

    But, We almost never have access to the true data generating process so
we can never know for sure if the model family being estimated includes
the generating process or not

    controlling the complexity of the model is not a simple matter of
finding the model of the right size, with the right number of parameters
Instead, we might find that the best fitting model (in the sense of
minimizing generalization error) is a large model that has been
regularized appropriately.

    \hypertarget{parameter-norm-penalties}{%
\subsubsection{7.1 Parameter Norm
Penalties}\label{parameter-norm-penalties}}

    Many regularization approaches are based on limiting the capacity of
models

    \hypertarget{note-that-for-neural-networks-we-typically-choose-to-use-a-parameter-norm-penalty-that-ux3c9-penalizes-of-the-affine-transformation-at-each-layer-and-leaves-only-the-weights-the-biases-unregularized.-each-bias-controls-only-a-single-variable.-this-means-that-we-do-not-induce-too-much-variance-by-leaving-the-biases-unregularized.-also-regularizing-the-bias-parameters-can-introduce-a-significant-amount-of-underfitting}{%
\subparagraph{Note that for neural networks, we typically choose to use
a parameter norm penalty that Ω penalizes of the affine transformation
at each layer and leaves only the weights the biases unregularized. Each
bias controls only a single variable. This means that we do not induce
too much variance by leaving the biases unregularized. Also,
regularizing the bias parameters can introduce a significant amount of
underfitting}\label{note-that-for-neural-networks-we-typically-choose-to-use-a-parameter-norm-penalty-that-ux3c9-penalizes-of-the-affine-transformation-at-each-layer-and-leaves-only-the-weights-the-biases-unregularized.-each-bias-controls-only-a-single-variable.-this-means-that-we-do-not-induce-too-much-variance-by-leaving-the-biases-unregularized.-also-regularizing-the-bias-parameters-can-introduce-a-significant-amount-of-underfitting}}

    it is still reasonable to use the same weight decay at all layers just
to reduce the size of search space.

    \hypertarget{l2-parameter-regularization}{%
\subsubsection{\texorpdfstring{7.1.1 \(\;L^2\) Parameter
Regularization}{7.1.1 \textbackslash{};L\^{}2 Parameter Regularization}}\label{l2-parameter-regularization}}

    a.k.a. weight decay, ridge regression, Tikhonov regularization This
strategy drives the weights closer to the origin by adding a
regularization term
\({\Omega}({\Theta})\; = \; \frac{1}{2}\left\Vert{\omega}\right\Vert^2_2\)

    \$ \omega ; = ; Q(A;+\alpha I)\textsuperscript{\{-1\}AQ}T\omega\^{}* \$

    effect of weight decay is to rescale \(\omega\) along the axes defined
by the eigenvectors of \$H;=;QAQ\^{}T \$

    Specifically, the component of \(\omega\) that is alogned with the
\(i\)-th eigenvector of \(H\) is rescaled by a factor of
\(\frac{\lambda_i}{\lambda_i + \alpha}\)

    Along the direction where eigenvalues of \(H\) are relatively large (ie.
\(\lambda_i >> \alpha\) ), effect of regularization is relatively small.
However, components with \$\lambda\_i \textless{}\textless{} \alpha \$
will be shrunk to have nearly zero magnitude

    Only the directions along which the parameters contribute significantly
to reducing the objective function are preserved relatively intact. In
directions that do NOT contribute to reducing the objective function, a
small eigenvalue of the Hessian tells us that movement in this direction
will not significantly increase the gradient.

    \hypertarget{l1-regularization}{%
\subsubsection{\texorpdfstring{7.1.2 \(\;L^1\)
Regularization}{7.1.2 \textbackslash{};L\^{}1 Regularization}}\label{l1-regularization}}

    \(\Omega (\theta) \; = \; \left\Vert{\omega}\right\Vert_1 \; = \sum_{i} \left\vert {\omega_i}\right\vert\)

    In comparison to \(L^2\) regularization, \(L^1\) regularization results
in a solution that is more sparse

    The sparsity property induced by \(L^1\) regularization has been used
extensively as a feature selection mechanism LASSO (Least Absolute
Shrinkage and Selection Operator) model integrates an \(L^1\) penalty
with a linear model and a least squares cost function

    \(L^2\) \textasciitilde{} MAP Bayesian inference with a Guassian prior
on the weights

    \(L^1\) \textasciitilde{} MAP Bayesian inference with an isotropic
Laplace distribution on the weights

    \hypertarget{norm-penalties-as-constrained-optimization}{%
\subsubsection{7.2 Norm Penalties as Constrained
Optimization}\label{norm-penalties-as-constrained-optimization}}

    Consider the cost function regularized by a parameter norm penalty 

    \$ \hat{J}(\theta;X,y) ; = J(\theta;X,y) ; + \alpha\Omega(\theta) \$

    Can Minimize a function subject to Constraints by constructing a
generalized Lagrange function consisting of the original objective
function + a set of penalties 

    \begin{itemize}
\tightlist
\item
  Each penalty is a product between a Coefficient (KKT multiplier) and a
  function representing whether the constraint is satisfied
\end{itemize}

    \begin{itemize}
\tightlist
\item
  If to constrain \(\Omega(\theta)\) to be less than some constant
  \(k\), a generalized Lagrange function: 
\end{itemize}

\begin{equation*}
\mathcal{L}(\theta,\alpha;X,y)\; = \; J(\theta; X,y)\; + \; \alpha(\Omega(\theta)\; - k)
\end{equation*}

    \begin{equation*}
\theta^* = 
\underset{\theta}{\text{arg min}} \;
\underset{\alpha,\alpha \geq 0}{\text{max}}\;{\mathcal{L}(\theta,\alpha)}
\end{equation*}

    solving this problem requires modifying both \(\theta\) and \(\alpha\) -
\(\alpha\) must increase whenever \(\Omega(\theta) > k\) - \(\alpha\)
must decrease whenever \(\Omega(\theta) < k\) - All positive \(\alpha\)
encourage \(\Omega(\theta)\) to shrink - Optimal value \(\alpha^*\) will
encourage \(\Omega(\theta)\) to shrink, but not so strongly to make
\(\Omega(\theta) < k\)

    Can also use explicit constraints rather than penalties

    \begin{itemize}
\tightlist
\item
  can modify SGD to take a step downhill on \(J(\theta)\) and then
  project \(\theta\) back to the nearest point that satisfies
  \$\Omega(\theta) \textless{} k \$.
\item
  useful if we know what value of \(k\) is appropriate and don't want to
  waste time searching for \(\alpha\) that corresponds to this \(k\)
\item
  penalties can cause non-convex optimization procedures to get stuck in
  local minima corresponding to small \(\theta\)
\item
  Explicit constraints implemented by re-projection only have an effect
  when the weights become large and attempt to leave the constraint
  region
\item
  Explicit constraints with reprojection imposes some stability on the
  optimization procedure - prevents positive feedback loop from
  continuing to increase the magnitude of the weights without bound.
\item
  In practice, column norm limitation is always implemented as an
  explicit constraint with reprojection so as to prevent any one hidden
  unit from having very large weights.
\item
  If we converted this constraint into a penalty in a Lagrange function,
  it would be similar to \(L^2\) weight decay but with a separate KKT
  multiplier for the weights of each hidden unit
\end{itemize}

    \hypertarget{regularization-and-under-constrained-problems}{%
\subsubsection{7.3 Regularization and Under-Constrained
Problems}\label{regularization-and-under-constrained-problems}}

    Linear Regression, PCA and many other Linear models depend on inverting
the matrix \(X^{T}X\)

    This is not possible whenever \(X^{T}X\) is singular - when data
generating distribution has no variance in some direction - when no
variance is observed in some direction - b/c fewer samples than features
\(\rightarrow\) Instead invert \(X^{T}X + \alpha I\) This regularized
matrix is guaranteed to be invertible

    When relevant matrix is invertible, closed form solutions exist Problem
with no closed form solution can be underdetermined eg. Logistic
regression where classes are linearly separable and \(w\) is able to
achieve perfect classification - then \(2w\) will also achieve perfect
classification and higher likelihood SGD wll continually increase the
magnitude of \(w\) until numerical overflow occurs Regularization will
cause SGD to quit increasing the magnitude of the weights

    (7.17) \begin{equation*}
w = (X^{T} X + \alpha I)^{-1} X^{T} y
\end{equation*}

    (7.29) - Definition of pseudoinverse \(X^+\) of a matrix \(X\)
\begin{equation*}
X^+ = \underset{\alpha\searrow0} {\text{lim}}\: (X^{T}X + \alpha I)^{-1} X^{T}
\end{equation*}

    (7.29) is the limit of (7.17) as the regularization coefficient shrinks
to zero

    \hypertarget{dataset-augmentation}{%
\subsubsection{7.4 Dataset Augmentation}\label{dataset-augmentation}}

    effective/easiest for classification - object recognition - speech
recognition

    Input Noise - can also bee seen as a form of data augmentation Dropout -
can be seen as constructing new inputs by multiplying by noise

    \hypertarget{noise-robustness}{%
\subsubsection{7.5 Noise Robustness}\label{noise-robustness}}

    In general, noise injection can be more powerful than simply shrinking
parameters, especially when noise is added to the hidden units
\(\rightarrow\) dropout

    Another way: adding noise to the weights - RNN - can be interpreted as a
Stochastic implementation of Bayesian inference over the weights -
Bayesian = model weights are uncertain and representable via a
probability distribution - Adding noise to the weights = practical,
stochastic way to reflect this uncertainty - Can also be interpreted as
pushing the model into regions where the model is relatively insensitive
to small variations in the weights, finding minima surrounded by flat
regions

    \hypertarget{injecting-noise-at-the-output-targets}{%
\paragraph{7.5.1 Injecting Noise at the Output
Targets}\label{injecting-noise-at-the-output-targets}}

    Most datasets have some amount of mistakes in the \(y\) labels
\(\rightarrow\) explicitly model the noise on the labels

    . incorporate into the cost function \(\rightarrow\) label smoothing

    Maximum likelihood learning with a softmax classifier and hard targets
may actually never converge---the softmax can never predict a
probability of exactly 0 or 1 exactly, so it will continue to learn
larger and larger weights, making more extreme predictions forever

    \hypertarget{semi-supervised-learning}{%
\subsubsection{7.6 Semi-Supervised
Learning}\label{semi-supervised-learning}}

    \begin{enumerate}
\def\labelenumi{\arabic{enumi}.}
\tightlist
\item
  Unlabled Examples from \(P(x)\)
\item
  Labeled Examples from \(P(x,y)\) are used to estimate \(P(y|x)\)
\end{enumerate}

    Goal is to learn a representation \(h = f(x)\) so that examples from the
same class have similar representations

    Construct models in which a generative model of either \(P(x)\) or
\(P(x,y)\) shares parameters with a discriminative model of \(P(y | x)\)

    Then, trade-off the supervised critereon \(- log P(y | x)\) with
unsupervised/generative critereon \(-log P(x)\; or\; -logP(x,y)\)

    Generative critereon then expresses a particular form of prior belief
about the solution to the supervised learning problem

    


    % Add a bibliography block to the postdoc
    
    
    
    \end{document}
